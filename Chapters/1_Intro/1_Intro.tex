%% WEEK 1

\chapter{A Recap of Undergraduate Set Theory} \label{Ch1:CH}
\thispagestyle{empty}

Set Theory can be viewed as a field of mathematics that encompasses all other mathematics. We can view set theory as beginning at the empty set $\emptyset$, also denoted $V_0$. We can construct subsequent sets $V_1, V_2, V_3, \ldots$ by taking power sets, ie, $V_1 = \Powset\of{V_0}$, $V_2 = \Powset\of{V_1}$, and so on. We can then define $V_{\omega}$ to be the union of all $V_n$, where $n < \omega$. We can then define $V_{\omega + 1} = \Powset\of{\omega}$, and so on. All of analysis, the field of mathematics Davoud Cheraghi once described as the ``rigorous study of infinite constructions'', is essentially about $V_{\omega + 1}$. In this way, set theory encompasses analysis, for example.

We call the hierarchy
\begin{align*}
    V_0 \subset V_1 \subset \cdots \subset V_{\omega} \subset V_{\omega + 1} \subset \cdots
\end{align*}
the \textbf{universe of sets}. Indeed, this is often represented diagrammatically as a $V$ that contains each of these $V$s, with $V_0$ at the bottom, $V_1$ above $V_0$, and so on.

There is a problem with the universe of sets: \textit{the power set operation is too strong}. That is, when we take the power set of a set, it ``throws too many elements in''. This leads to questions like the \textit{continuum hypothesis}, which asks the question of whether $\aleph_0 = \abs{\R} = \abs{V_{\omega + 1}}$ is equal to $\aleph_1$.

Something we will study quite closely in this course is a slightly different version of this universe, known as \textbf{Gödel's Constructible Universe}. This is a universe of sets that is not built up from $\emptyset$ using the power set, but using a \textit{different} operation $\Defin$, defined as follows: for a set $A$, viewing it as a first-order structure $(A; \in)$,
\begin{align*}
    \Defin(A) &:= \setst{x}{x \text{ is a 1st-order definable structure over } (A; \in) \text{ using parameters from } A}
\end{align*}
We can show that for finite sets, $\Defin$ agrees with $\Powset$. Moreover, we can show that for a countable set $A$, $\Defin(A)$ is countable as well. The point is that if we then define a universe
\begin{align*}
    \parenth{L_0 = \emptyset} \subset \parenth{L_1 = \Defin\of{L_0}} \subset \cdots \subset \parenth{L_{\omega} = \bigcup_{n < \omega} L_n} \subset \cdots
\end{align*}
we not only have agreement between $L_n$ and $V_n$ for all $n \leq \omega$, but we can build a model of ZFC in which the continuum hypothesis is true!

This is something we will explore in great detail in this course.

We will begin by reviewing the ZFC axioms.

\section{The Zermelo-Fraenkel Axioms and the Axiom of Choice}

We begin by reviewing the Zermelo-Fraenkel Axioms of Set Theory. We work in a first-order language with only one relation symbol, denoted $\in$. In principle, we would want to distinguish between the symbol in the language and its interpretation in any model, but we will not do this in practice.

We begin with the Axiom of Extensionality.

\begin{baxiom}[The Axiom of Extensionality]\label{ZFC:Ext}
    \begin{align*}
        \forall A \, \forall B \, \brac{\parenth{\forall x \parenth{x \in A \lr x \in B}} \to \parenth{A = B}}
    \end{align*}
\end{baxiom}

We define the $\subseteq$ symbol, used infix as $A \subseteq B$, to be shorthand for the formula $\forall x \, \parenth{x \in A \to x \in B}$. \Cref{ZFC:Ext} tells us that
\begin{align*}
    \forall A \forall B \parenth{\parenth{A \subseteq B \land B \subseteq A} \to A = B}
\end{align*}

We now define the Axiom Scheme of Comprehension.

\begin{baxiom}[The Axiom (Scheme) of Comprehension]\label{ZFC:Compr}
    Let $x$ and $y$ be free variables. For all formulae $\varphi\of{x, y}$, the following is an axiom:
    \begin{align*}
        \forall A \forall y \exists B \forall x \brac{x \in B \lr \parenth{x \in A \land \varphi\of{x, y}}}
    \end{align*}
    Intuitively, this means we can define
    \begin{align*}
        B = \setst{x \in A}{\varphi\of{x, y}}
    \end{align*}
    This is an axiom scheme because we can increase the arity of $\varphi$ and have more free variables $y_1, y_2, y_3, \ldots, y_n$ for any $n$.
\end{baxiom}

Note that \Cref{ZFC:Ext} tells us that we can replace the existential quantifier for $B$ in \Cref{ZFC:Compr} with an existence and uniqueness quantifier without changing its meaning.

Before proceeding further, we define the empty set. First, observe that in first order logic, structures are required to have non-empty universes. Thus, for every structure $\parenth{A, E}$ in the language of set theory, with $E \subseteq \parenth{A \times A}$ representing equality, we must have
\begin{align*}
    \parenth{A, E} \models \exists x \parenth{x = x}
\end{align*}
The Completeness Theorem then tells us that
\begin{align*}
    \vdash \exists x \parenth{x = x}
\end{align*}
ie, that the formula $\exists x \parenth{x = x}$ \textit{must be a theorem} in the language of sets. Hence, the axioms \Cref{ZFC:Ext,ZFC:Compr} tell us that
\begin{align*}
    \vdash \exists! A \, \forall x \, \parenth{x \notin A}
\end{align*}
In other words, it is a theorem in the language of sets that there is a unique set that contains no members whatsoever. This unique set is denoted $\emptyset$, and is called the \textbf{empty set}. Its existence (and uniqueness) is not a distinct axiom, but a direct consequences of \Cref{ZFC:Ext,ZFC:Compr}.

Next, we give the axiom of pairing.

\begin{baxiom}[The Axiom of Pairing]\label{ZFC:Pairing}
    \begin{align*}
        \forall x \, \forall y \, \exists A \, \parenth{x \in A \land y \in A}
    \end{align*}
\end{baxiom}

\Cref{ZFC:Pairing} essentially gives us a way of constructing, for any $x$ and $y$, the unique set $A = \set{x, y}$ with the property that
\begin{align*}
    \forall z \parenth{z \in A \lr \parenth{z = x} \lor \parenth{z = y}}
\end{align*}
In particular, it allows us to construct, for any $x$, the set $\set{x, x}$, which, by \Cref{ZFC:Ext}, is exactly $\set{x}$. In other words, it tells us that we can stick any $x$ into a set that only contains $x$.

In a similar flavour, we can construct a \textit{union} of any family of sets. Note that this is distinct from pairing, because to construct $\set{x, y}$ as a union of $\set{x}$ and $\set{y}$, one needs to know that one can construct $\set{x}$ and $\set{y}$ in the first place. This requires pairing (or some version of it).

Now, we starte the axiom of unions.

\begin{baxiom}[The Axiom of Unions]
    If we denote by $\calF$ a family of sets,
    \begin{align*}
        \forall \calF \, \exists B \, \forall A \, \brac{A \in \calF \to A \subseteq B}
    \end{align*}
\end{baxiom}

As usual, we can use \Cref{ZFC:Ext,ZFC:Compr} to show that
\begin{align*}
    \bigcup \calF = \setst{x}{\exists A \subseteq \calF \, \parenth{x \in A}}
\end{align*}
is well (and uniquely) defined.

Next comes the power set axiom.

\begin{baxiom}[The Axiom of Power Sets]
    \begin{align*}
        \forall A \, \exists \calF \, \forall X \, \brac{X \subseteq A \to X \in \calF}
    \end{align*}
\end{baxiom}

Again, \Cref{ZFC:Ext,ZFC:Compr} tell us that this is equivalent to
\begin{align*}
    \forall A \, \exists! \calF \, \forall X \, \parenth{X \in \calF \lr X \subseteq A}
\end{align*}
This justifies the definition
\begin{align*}
    \Powset(A) := \setst{X}{X \subseteq A}
\end{align*}

Next comes the second (and last) axiom scheme in ZFC, the axiom scheme of replacement.

\begin{baxiom}[The Axiom (Scheme) of Replacement]\label{ZFC:Replacement}
    For each formula $\varphi\of{x, y, z}$ that does not contain $B$ as a free variable, the following is an axiom:
    \begin{align*}
        \forall A \, \forall z \, \brac{
            \brac{\forall x \in A \, \exists! y \, \varphi\of{x, y, z}}
            \to
            \brac{\brac{\exists B \, \forall x \in A \, \exists y \in B \, \varphi\of{x, y, z}}}
        }
    \end{align*}
    As in \Cref{ZFC:Compr}, we can introduce more free variables in lieu of $z$.
\end{baxiom}

We can unpack this slightly complicated formula using \Cref{ZFC:Ext,ZFC:Compr}: it essentially tells us that we can define
\begin{align*}
    \setst{y}{\exists x \in A \, \varphi\of{x, y, z}}
\end{align*}
The quantifiers $\forall A$ and $\forall z$ merely introduce variables. What the rest of the axiom (scheme) tells us is that \textbf{if} for every $x$, there is a unique \textit{witness} $y$ for which the formula $\varphi\of{x, y, z}$ holds, \textbf{then} we can collect all these witnesses into a set that is contained $B$, ranging over $x \in A$. Intuitively, ``the number of such witnesses is \textit{small enough} that it \textit{can be collected into a set}.''

Indeed, if it looks like $\varphi\of{x, y, z}$ defines a function with domain $A$, then it really does! If we can define $A \times B = \setst{\parenth{x, y}}{x \in A \text{ and } y \in B}$ (which we have yet to show we can do, but let's say we can), then we can put
\begin{align*}
    f = \setst{\parenth{x, y} \in A \times B}{\varphi\of{x, y, z}}
\end{align*}
which is a set by \Cref{ZFC:Compr}. Effectively, \Cref{ZFC:Replacement} tells us that \textit{if the domain of a function is a set, then so is its range}: everything in its ``image'' (a \textit{witness}) can be collected into some set. Of course, we only include this comment for the sake of intuition; as far as the formalism goes, we cannot talk about domains and ranges of functions yet, because functions are not yet defined. That being said, \Cref{ZFC:Replacement} is an essential step in the process of actually \textit{defining} functions (or showing that this is something we can \textit{do}).

One situation in which \Cref{ZFC:Replacement} is used (and will be used once sufficiently many axioms are stated) is in the situation we talked about at the start of \Cref{Ch1:CH}. We'll see that there is a formula $\varphi\of{n, S}$ so that $\varphi\of{n, S} \lr \parenth{n \in \omega \land S = V_n}$. Then, \Cref{ZFC:Replacement} would imply that
\begin{align*}
    \grpres{V_n}{n < \omega}
\end{align*}
really is a sequence. This allows us to define $V_\omega = \bigcup_{n < \omega} V_n$ and continue.

% Next time, we will look at the Foundation Axiom.
\input{Chapters/1_Intro/1_2_Another_Section.tex}